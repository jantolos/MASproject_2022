% !TeX spellcheck = en_US
\documentclass{llncs}
%
\usepackage{url}
\usepackage[croatian]{babel}
% \usepackage[english]{babel} % odkomentirati ovo za pisanje na engleskom jeziku
\usepackage[utf8]{inputenc}
\usepackage[T1]{fontenc}
\usepackage{csquotes}
\usepackage{graphicx}
%
\begin{document}
	\title{Naslov završnog ili diplomskog rada}
	\author{Ime i prezime autora/autorice}
	\podaci{
	    Naziv studija (puni naziv)
	}{
	    Razina studija (puni naziv)
	}{
	    email@foi.unizg.hr
	}
	
	\maketitle
	
	\begin{abstract}
		Opsega do 100. Ovaj sažetak opisuje ukratko čime se rad bavi te glavne teze i smjer rada.
	\end{abstract}
	
	\section{Uvod}
	
	Radni nacrt završnog ili diplomskog rada je priprema za daljnju razradu teme. Radni nacrt mora sadržavati \cite{oraictolic2011AkademskoPismoStrategije}:
	\begin{itemize}
	    \item naslov teme;
	    \item prethodni sadržaj -- što već postoji, što ste već napravili (ako je primjenjivo) i slično;
	    \item hipotetični uvod -- postavljanje problema, osnovni teorijski pregled te osnovna argumentacija, predviđeno korištene metodološke postupke;
	    \item ciljeve rada -- što planirate napraviti, pregledati, istražiti;
	    \item preliminarni zaključak;
	    \item radnu bibliografiju -- kratki popis literature koja (predviđeno) čini osnovu za izradu rada, a koju referencirate u sadržaju radnog nacrta \cite{Russell2010AImodern,okresaduric2019ModellingFormingTemporary}.
	\end{itemize}
	
	
	\section{Radni nacrt}
	
	Radni nacrt opisuje važne korake koje predviđate da ćete poduzeti u tijeku izrade svog rada. To je skup Vaših bitnih ideja i prikupljenih argumenata koji su u funkciji glavne ideje rada. Argumenti mogu biti afirmativni i negativni, tj. podupirati osnovnu ideju ili ne.
	
	Ako Vam je jasna osnovna ideja, radni nacrt služi za dodatnu razradu te ideje, za stvaranje vremenskog i tematskog plana pisanja i realizacije rada.
	
	Ako Vam osnovna ideja rada nije potpuno jasna, radnim nacrtom proširite i produbite svoju potragu za izvorima, kako biste ideju razbistrili.
	
	\section{Ostalo}
	
	Radni nacrt može biti duljine od najviše dvije stranice te time predstavlja i vježbu za sažimanje ideja i njihovo predstavljanje u kompaktnom, ali visoko informativnom obliku. Prijedlog: iskoristite obje stranice, a sadržajem \textit{prodajte} i najavite svoj rad.
	
	Dodatni primjeri korištenja \LaTeX naredbi mogu se pronaći u dokumentu \lstinline+Rad.tex+.
	
	\section{Nova sekcija}
	
	\lipsum[1-2]

	\printbibliography
\end{document}